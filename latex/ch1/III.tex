美国人战后想得到点什么呢?毫无疑问也是安全,但是与斯大林不同,他们还无法肯定做些什么才能得到它。究其原因跟二战将美国置于一个尴尬的境地有关:对美国来说,在与其他地方相对隔离的情况下作为世界榜样的时代一去不返了。

纵观历史,美国人一直在试图这么做。他们不必担心自身安全问题,因为大洋将他们与有可能伤害他们的国家隔离开来。正如Thomas Paine在1776年的预言,“一个小岛可以永久地统治一个大陆”这简直是难以令人置信的,正是这一点使美国能从大不列颠英国独立出来。尽管海军力量非常强大,英国人从来也无力将其海军军力扩展到三千海里的海域之外,以将美国人留在帝国之内,或者阻止他们占领北美大陆。指望其他欧洲国家能做到这一点更是天方夜谭,因为伦敦的继任政府已经与美国达成协议,同意在西半球不再保留殖民存在。因此,美国享有一种特殊优待,即在保持大块势力范围的同时不必担心会侵犯其他大国的利益。

在思想领域美国的确也曾需求国际影响:要知道他们的独立宣言比人人生而平等这一激进主张更进一步。但是在独立后的140年间,他们并没有努力去实现他们的主张。美国只想做出个榜样;至于如何学习这个榜样或者在何种情况下学习这个榜样是其他国家自己的事情。”她满心憧憬着一切自由和独立,” 国务卿约翰 quincy 亚当斯在1821年宣布说,但“她只捍卫和维护她自身。”因此,尽管秉承着国际主义意识形态,美国人实践的确是孤立主义:这个国家还没有得出为了自身安全需要道德标准移植的结论。它的外交和军事政策的野心相对来说并不大,还远未达到人们对一个如此规模和实力国家的预期。

直到第一次世界大战美国人才打破这个模式。由于担心帝国主义德国可能会击败英国和法国,武德罗 威尔逊游用恢复欧洲力量平衡需要美国军队的参与这一理由来游说他的子民——他甚至用意识形态术语来力证这一地缘政治目标。他坚持这个世界必须得到“民主所需要的安全”。威尔逊还建议,作为和平安排的基础,应该成立国际联盟这一组织,主要目的是在国家间也能推行类似于法治精神这类东西,就像国家——至少是启蒙国家——在人民间推行这些东西一样。The idea that might alone makes right would, he hoped, disappear.

然而,威尔逊的愿景和重新恢复的平衡都被证明是不成熟的。一战的胜利并未使美国称为一个世界性强国;相反它使美国人认识到了过度承诺的危险性。威尔逊关于战后集体主义安全组织的计划已经超过了美国人民的预期。

在战时,罗斯福有他的四个优先。首先是支持盟友——主要是英国,苏联和(不太成功支持的)中国国民党——因为找不到其他办法获胜:美国无法独自去的抵挡德国和日本。第二是在战后安排上保证盟友间和谐,不如此的话战后和平的前景将堪忧。第三个优先与战后安排的本质有关。罗斯福期望他的盟友能支持一个能最大程度限制战争的方案。这意味着一个有权震慑甚或惩罚贪婪的新集体安全组织,和一个能防止经济衰退的新全球性经济组织。最后,这个安排还必须能推销给美国人民:罗斯福不想重复威尔逊的错误而把这个国家带离它准备去的方向。战后已无法恢复孤立主义,但是美国也没准备让战后世界成为战前的一个翻版。

最后,简单谈谈英国的目标。正如丘吉尔所说,他们的目标要简单的多:尽一切可能存活, 即使这意味着将盎格鲁——美国联盟的领导权拱手相让给美国,即使这意味着削弱大英帝国,即使这也意味着要与苏联合作,要知道丘吉尔早先曾经乐见这个布尔什维克革命后的政权倒台。英国将尽可能去影响美国人——他们立志做一回希腊人,当罗马人的导师——不过无论何时他们都不会和美国人翻脸。斯大林更期望一个独立的英国,能抗拒美国甚至对美国动武,但是这对那些实际影响了战时英国和战后重要战略的人来说简直太不可思议了。