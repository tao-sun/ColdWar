那斯大林到底想要些什么呢?先谈谈他是非常有意义的,在战后的三巨头中只有他能在保持权威的同时有时间来考虑事情的轻重缓急。斯大林在战后已经65岁了,他身心疲惫,虽然周围被谄媚者包围,内心却非常孤独——但仍然牢牢执掌着权柄。 他留着短髭,牙齿发黄,满脸麻子,眼珠是黄色的,一个美国外交官曾回忆到,”给我的印象象一个老迈的伤痕累累的猛虎… 普通人简直无法想象在他朴实无华外表下隐藏的精于算计,勃勃野心,对权利的贪恋,妒忌、冷酷和狡诈”。通过三十年代一系列的大清洗,斯大林早已排除了异己。随从们心理都明白,他一皱眉或一抬手都可能意味着生死。这个只有五英尺四英寸的大腹便便的矮个子巨人正把自己凌驾于一个大国之上。

斯大林战后目标依次为自身安全、政权、国家和意识形态。他设法确保在内部没人再能破坏他的规则,在外部没人再能威胁到他的国家的安全。世界其他地方的共产主义的利益,虽然也是会令人钦佩的事情,无论如何也无法和他一手建立起来的苏维埃政权相比。斯大林集自恋、偏执和绝对权威于一身:无论在苏联还是在国际共运中,他都是让人又敬又畏的人物。

斯大林也有痛苦纠结的地方。战争中过大的损失给了苏联在战后攫取更大利益的借口,但是这个国家业已被战争洗劫一空,仅靠自己的力量已经无法保全这些到手的利益。苏联需要和平、经济援助以及前盟国的外交默许。它甚至别无选择只能跟美英两国合作:就像他们要依靠斯大林来击败希特勒一样,斯大林如果想以合理的代价达到战后的目标,也要仰赖盎格鲁-美国的友善和青睐。因此他既不想要一场热战也不想要一场冷战。至于他是否会驾轻就熟地避开这些那就是另外一回事了。

斯大林对战时盟友以及他们的战后目标的理解更多是基于一厢情愿,而不是华盛顿和伦敦所乐见的基于对优先级的正确评估。在这一点上马克思列宁主义意识形态对斯大林的影响很大,因为他的错觉皆源于此。其中最重要的一点是来自列宁主义,即相信资本主义之间的合作都不会长久。它们本质上的贪婪将迟早占上峰,因为将利润置于政治之上的冲动往往是不可抗拒的,这也使得共产主义者们只需要耐心等待敌人的自我毁灭即可。“ 我们和资本主义中民主部分的结盟最终取得了胜利,因为阻止希特勒占领也是他们的利益之所在”在战争快结束的时候斯大林这样说道。 “今后我们也将和这部分资本主义决裂。”

资本主义存在危机的说法貌似是合理的。毕竟第一次世界大战是一场资本主义之间的战争;它为世界上第一个共产主义国家的建立提供了契机。战后尚存的资本主义国家在大萧条的时候也是只能自顾而无暇合作去挽救全球经济或者维持战后安排:纳粹德国趁势而起。斯大林相信随着第二次世界大战的结束,经济危机一定会重来。资本主义因此会需要苏联,而不是苏联需要资本主义。这也是为什么他会满心期待美国借给苏联几十亿美元用以重建的原因:因为不如此的话,美国在即将到来的全球危机中将无法找到市场去贩卖它的产品。

紧接着另一个资本主义超级大国,英国—— 斯大林一直低估了其衰落程度——跟美国由于经济上的对抗而翻脸也是迟早的事情 : “资本主义国家间的战争仍然是不可避免的,”直到1952年他还坚持这么认为。在斯大林看来,历史总有一天会补偿苏联在二战中所遭受的灾难。为了达到他的目标,跟英美正面的冲突是没有必要的。他只需要等着资本主义国家间的内斗开始,到那时候恶心的欧洲人就会自己来拥抱共产主义这根救命稻草。

因此,斯大林的目标不是要重建欧洲的平衡,而是象希特勒那样谋求占领整个欧洲。1947年他曾不无惋惜地袒露道 “如果丘吉尔在法国北部开辟第二战场地时间再晚一年的话,红军就能打到法国…… 我们曾把玩过占领巴黎的想法”。但是与希特勒不同的是,斯大林并没有制定一个详尽地时间表。他热烈欢迎诺曼底登陆,尽管这次行动基本排除了红军不久就打到西欧的可能性:毕竟打败德国才是最重要的。他也从来没有忽略外交途径的努力,尤其是他至少在某一段时间曾期望通过与美国合作来达到目的。真难想象罗斯福如果未曾表明美国会避免在欧洲寻求自己的势力范围。因此,斯大林的这个设想堪称宏大:历史性地通过和平手段取得欧洲的统治权。但这个设想也并非没有瑕疵,因为它并没有考虑战后美国的目标也是不断变化的。