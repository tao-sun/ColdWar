欢笑、握手、舞蹈、觥筹交错、憧憬未来,战后此种欢庆场面随处可见。1945年4月25日,两只军队从两个相反方向攻入易北河流经的东德城市Torgau,实现了首次胜利会师,纳粹德国就此一分为二。五天之后,Adolf Hitler 一枪将自己的脑浆迸射到满是瓦砾的柏林城下。一周以后, 德国宣布无条件投降。而获胜的盟军统帅富兰克林 罗斯福、Winston 丘吉尔、Josef 斯大林早将把酒言欢和筹划未来的时刻提前到了1943年11月的德黑兰会议和1945年2月的雅尔塔会议。倘若他们统帅的军队在其真正用武之地的前线战场上未能狂奔着庆祝胜利的话,这一切真是毫无意义。

然而为什么两只军队的相会如此地充满戒备,象对待来自外星得访客一般?为什么他们见到对方和自己如出一辙,会如此的惊讶但马上又感到宽慰?尽管如此,他们各自的司令官还是坚持要分开受降,西线仪式安排在法兰西的Reims,于五月七号举行,东线的于五月八号在柏林举行,这又如何解释呢?为什么苏维埃当局试图瓦解莫斯科在正式宣布德国有条件投降后自发进行的反美示威呢?为什么美国在一周以后突然叫停了根据租借法案向苏联运送重要物资?为什么罗斯福的的重要助手,在1941年一手促成盟军组建的Harry Hopkins,在罗斯福死后六周匆忙赶到莫斯科试图阻止盟国的崩溃?为此,几年后丘吉尔将回忆录命名为《胜利与悲剧》,这是为什么呢?

这一切的答案只有一个:另外一场战争——一场非军事的意识形态和地缘政治之争——已经在获胜盟国的主要成员间上演。尽管盟军在1945年的春天胜利了,它的成功只是有赖于水火不容的两个体系间对共存目标的追求。悲剧就在于胜利要求胜利者们要么停止做以前的自己,要么放弃很多只要打赢战争才能实现的愿景。