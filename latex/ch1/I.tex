如果不知情的游客在1945年春天来到易北河畔,他们眼中的俄美军队可能没什么不同,这正如俄美社会在表面上的相似。美利坚合众国和苏维埃联盟都在革命的隆隆炮火中诞生;他们对自身意识形态的普适性都深信不疑,其领袖们都认为在自己地盘上的东西拿到其他地方也照样行得通;两个洲际国家的疆域都非常辽阔:他们当时分别是世界上面积第一和第三的国家;他们进入二战的方式都如此相象:1941年6月22日德国闪电入侵苏联,同年的12月7日日本偷袭 珍珠港,四天后希特勒借此对美宣战。然而,相似仅止于此。他们之间的差别实际是巨大的,这一点在地球上人尽皆知。

发生在二战之前一个半世纪的美国革命体现了对中央集权的深深不信任。立国的先贤们坚信受约束的权利才会带来自由和公平。一部天才的宪法, 在地理上与可能的对手的远隔重洋,大自然赐予的丰富资源,这一切使美国成为了一个强大的国家,这些在二战中得到了集中的反映。然而美国的成功是建立在严格限制政府对日常生活控制的之上的,而限制的实现有赖于思想的传播以及经济和政治体制的建立。尽管奴隶主义的遗毒仍然存在,土著居民也近乎灭绝,种族、性别和社会等方面的歧视问题仍然在持续,美国人民还是可以在1945年自豪地宣布他们生活在地球上最自由的国度。

布尔什维克革命发生在二战之前四分之一个世纪,它推崇集权统治,并将其作为推翻阶级敌人的手段,最终建立起一个将无产阶级革命传遍世界大本营,它。卡尔 马克思在1948年的共产党宣言上宣称工业资本家的崛起和对无产阶级的剥削同时使无产阶级得以壮大,这一切将最终使无产阶级得以解放。不甘于等待的符拉基米尔 伊里奇 列宁在1917年加速了历史进程,开始统治俄国并强制推行马克思主义。实际上马克思曾预言革命只会在发达的工业社会发生,而俄国并不满足这一条件。继任者斯大林改变了这一状况,为了更好的适应马克思列宁主义意识形态,他对俄国进行了大规模变革,将一个几乎没有自由传统的庞大农业国转变为一个毫无自由的高度工业化的国家。结果,苏维埃社会主义共和国联盟在二战后成为世界上最专制的国家。

战胜国之间的反差如此巨大,其实他们在1941年到1945年的战争经历也是如此。美国多线作战,在太平洋的对手是日本,在欧洲的是德国,但是伤亡率却出奇的低:在所有战场上加起来只有不到30万美国人死亡。由于远离战场,除了起初的珍珠港偷袭外,美国本土并未遭受实质性的攻击。美国的盟友英国的情况也差不多(英国战时死亡大约为357000人)。他们都能选择战斗的地点、时间和环境,这大大降低了战争开支和风险。不同的是,美国经济通过战争得到极大发展:战时花销使得美国的GDP在四年内几乎翻了一倍。如果真有战争能被算作“好战争”的话,对于美国来说这场战争就应该算上一个。

苏联就没有那么好的运气。卫国战争虽是单线作战,但无疑是历史上最惨烈的战争之一。城市和村镇被蹂躏,工业要么被摧毁要么被迫迁往乌拉尔山以东地区,人们如果不投降就只有在战场上死命抵抗,而且战场的地形和环境都是自己无法控制的。苏联在战争中的平民和军人的伤亡数字的统计不准确是出了名的,但大概还是有2700万人直接死于这场战争,这大约是美国的90倍。战争的花费是巨大的,可以说这个国家在1945年已经千疮百孔,只是勉强得以保全。一个现代观察家曾回忆到:卫国战争的有“最令人可怕的战场,同时也是俄国人最值得骄傲的记忆”。

相对于战争承担的不对称性,战胜国在战后安排方面要公平的得多。美国并没有承诺要改变自己漠视欧洲事务的传统——罗斯福甚至在德黑兰向斯大林保证要在战后两年内撤回美国军队。有鉴于1930年代大萧条时期经济低迷程度之深,美国在当时也无法保证战时繁荣会继续,或者民主会在业已扎根几个国家之外开花结果。美国和英国如果没有斯大林的帮助是无法战胜希特勒的,这一严酷的事实意味着第二次世界大战仅仅是反法西斯的胜利——而不是反极权主义及其对未来影响的胜利。

苏联尽管遭受重大损失,它的优势还是非常明显的。它本身就是欧洲的一部分,军队自然可以不从欧洲撤离。它的指令性经济已被证明可以实现完全就业,而资本主义民主却在战前无法做到这一点。它的意识形态在欧洲受到广泛尊重因为那里的共产主义者在很大程度上领导了对德国人的抵抗。在战胜希特勒中红军承担了不成比例的负担,这在道义上也允许苏联对战后安排施加实质性影响,甚至占主导地位。另外,在那时很容易让人相信,极权的共产主义和民主的资本主义一道都是未来潮流。

苏联还有一个独特优势——它的领导人经历过战争考验,这在战胜国中是独一无二的。罗斯福在1945年4月12日逝世,一个没有经验的问题副总统Harry 杜鲁门入主了白宫。三个月后丘吉尔在英国大选中意外落败,软弱的Clement 艾德礼为英国首相。对比起来,斯大林从1929年就已经成为苏联无可争议的统治者,他重塑了这个国家并且领导它取得的二次大战的胜利。他狡诈、强硬但在任何场合下都能做到镇静果断,这个克里姆林宫的独裁者清楚地知道他在战后岁月中想要些什么。而杜鲁门和Attlee和他们的国家在这个事情上却是犹豫不决。